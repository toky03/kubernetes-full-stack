\subsection{Database}
In the further sections we will use a plain mysql docker container as a database. In order to use the Application completely locally, a local mySQL database has to be installed and a schema for the random generator application needs to be provided for a specific randgenuser.
If there is not yet a local mySQL database installed it can be done with the following command:
\begin{lstlisting}[language=Bash]
sudo apt-get update
sudo apt-get install mysql-server
\end{lstlisting}
After the installation log in to MySQL as root. 
\begin{lstlisting}[language=Bash]
sudo mysql
\end{lstlisting}
Now the Database and the user can be created and permissions to the respective schema will be given.
\begin{lstlisting}[language=SQL]
CREATE database db_example;
CREATE USER 'springuser'@'localhost' IDENTIFIED BY ThePassword;
GRANT ALL PRIVILEGES ON *.db_example TO 'springuser'@'localhost';
\end{lstlisting}
With this, the Database is ready to used by the Random Generator Example from localhost.

\subsection{Random Generator}
The Random Generator will be a verry simple Python App which has an unique Id per Service instance. It will return his id and a new Random Number. For this application we create a folder called RandGen with the following Structure.

\dirtree{%
.1 RandGen.
.2 Dockerfile.
.2 rand\_gen.py.
.2 requirements.txt.
}
The Dockerfile is needed to create the Docker Image which will be used from Kubernetes to Create the Random Generator Service.
rand\_gen.py is the complete Random Generator Python application based on Flask\footnote{More information about Flask can be found on the official Homepage \url{http://flask.pocoo.org/}}.
 \lstinputlisting[language=Python]{Applications/RandGen/rand_gen.py}
In requirements.txt are the Python packages specified. In this case it is flask. therefore this file contains only one line which is
\begin{verbatim}
Flask==1.0.2
\end{verbatim}

The Dockerfile to create the Docker image:


\subsection{Middle Tier}

The Middle Tier Application is created with Spring Initializer.
Unter the following link \url{https://start.spring.io/} Helps to bootstrap very fast a simple Spring Boot application.
In our case the fields should be filled out as followed:
\begin{tabbing}
\begin{tabular}{ll}
Project & Maven Project \\
Language & Java \\
Spring Boot & 2.1.3 (All versions would apply) \\
Group & ch.toky.randgen \\
Artifact & middletier \\
Name & MiddleTier \\
Description & Middle Tier Service for Random Generator Application \\
Packaging & Jar \\
Java Version & 8 \\
Dependencies & Web, JPA, MySQL, Rest Repositories
\end{tabular}
\end{tabbing}
The Website will create a zip file to download. This zipfile has to be extracted to the desired Project folder.
\dirtree{%
.1 middletier.
.2 Dockerfile.
.2 mvnw.
.2 mvnw.cmd.
.2 pom.xml.
.2 src.
}
all blue folders are already given. the Dockerfile still has to be created.

Now inside of the Project create the following Project Structure and files.

\dirtree{%
.1 ch.toky.randgen.middletier.
.2 MiddletierApplication.java.
.2 ch.toky.rand-gen.middle-tier.model.
.3 PodStat.java.
.3 RandomNumber.java.
.2 ch.toky.rand-gen.middle-tier.repository.
.3 PodStatRepository.java.
}
The file MiddleTierApplication.java should already be in place. as Spring Boot Initializr created this with the complete Folder structure. the packages controller, model and repository need to be created.
Inside of controller, A file named MiddletierController.java needs to be created with the following content.

Within the model package two new Files have to be created: PodStat.java and RandomNumber.java.

PodStat is an Entity and therefore the class has to be annotated with @Entity.
It contents the following fields with their respective getter and Setter Methods.
\begin{lstlisting}[language=Java]
@Id
@GeneratedValue(strategy=GenerationType.AUTO)
private Long podStatID;
private String id;
private Long timeStamp;
private Long counter;
\end{lstlisting}

The class RandomNumber is only a DTO which is retreived from the random generator Service. Therefore there is no need for a annotation.
Only the following Fields needs to be declared with their getters and setters:

\begin{lstlisting}[language=Java]
private String id;
private Long randNumber;
\end{lstlisting}

In the repository package the file PodStatRepository.java will be created.
As this java file is not a class but an interface it needs to be changed to interface and extends JpaRepository<PodStat, Long>
and it will be annotated with @Repository

Now those two methods are created 
\begin{lstlisting}[language=Java]
@Query("Select count(ps.id) from PodStat ps where ps.id = ?1")
Long countUniqueId(String id);

@Query("Select count(ps.id) from PodStat ps")
Long findMaxCount();
\end{lstlisting}

Witin the controller all the endpoints are declared. Therfore it will be annotaded with @RestController

Those fields are needed within the controller and are therefore declared first.
\begin{lstlisting}[language=Java]
@Autowired
private ObjectMapper objectMapper;
@Autowired
private PodStatRepository podStatRepository;
private Map<String, String> env = System.getenv();
\end{lstlisting}
As there should not be any hardcoded url according to the 12 Factor application \footnote{Accessible at \url{https://12factor.net/}}, the Url will be retreived through a System Variable.

Now a controller Method with the business logic can be declared:
\begin{lstlisting}[language=Java]
@RequestMapping(value = "/", produces = "application/json")
public RandomNumber getRandom() {

    RandomNumber randNum = getNewRandomNumber();
    Long maxCountId = podStatRepository.countUniqueId(randNum.getId());
    Long maxCountOverall = podStatRepository.findMaxCount();
    PodStat tmpPod = new PodStat();
    tmpPod.setId(randNum.getId());
    tmpPod.setTimeStamp(maxCountOverall+1);
    tmpPod.setCounter(maxCountId +1);
    podStatRepository.save(tmpPod);

    return randNum;

}
\end{lstlisting}
The above Method listens on the path / and returns the Random number after saving it to the database with the actual count.

Now the last Method is used to retreive the Random Number from the Random Generator Service.
\begin{lstlisting}[language=Java]
private RandomNumber getNewRandomNumber() {

    String randGenUrl=env.get("RANDOM_GENERATOR_URL");
    String URL = randGenUrl;
    RestTemplate restTemplate = new RestTemplate();

    return   restTemplate.getForObject(URL, RandomNumber.class);

}
\end{lstlisting}
Above Source code contains the Controller and the Service, which should be separated optimally.
Next two files represents POJOs which are going to be used from controller and provided by Repositories.

The last thing which needs to be done for the Middle Tier Service is to setup the default values for the Database Connection.
\begin{lstlisting}
spring.jpa.hibernate.ddl-auto=update
spring.datasource.username=random_user
spring.datasource.password=random_password
spring.datasource.url=jdbc:mysql://localhost:3306/rand_numbers
\end{lstlisting}
The datasource configurations are going to be overwritten by Environment Variables once they run inside of a Docker Container.

\begin{lstlisting}
FROM anapsix/alpine-java:latest

ADD target/middletier-0.0.1-SNAPSHOT.jar /opt/middle-tier.jar

ENV SPRING_DATASOURCE_URL=jdbc:mysql://rand-gen-database/rand_numbers

ENV SPRING_DATASOURCE_USERNAME=random_user

ENV SPRING_DATASOURCE_PASSWORD=random_password

ENV RAND_GEN_URL=rand-gen-app

CMD java -jar /opt/middle-tier.jar

\end{lstlisting}
We are using anapsix base image, as we want to keep the Docker image as small as possible.

Finally we can create the docker image with the following Command.

Important is, we need to be in the root directory of the Projecct where also the Dockerfile is stored.
\begin{lstlisting}[language=Bash]
docker build -t middle-tier .
\end{lstlisting}
the flag -t inicates that the name of the image and the dot at the end tells docker to use the Dockerfile from the current working directory.


\subsection{Statistic Service}
The Statistic Service has the same setup as the Middle Tier client.
Therefore the Project will be created too with Spring Boot initializer.
\begin{tabbing}
\begin{tabular}{ll}
Project & Maven Project \\
Language & Java \\
Spring Boot & 2.1.3 (All versions would apply) \\
Group & ch.toky.randgen \\
Artifact & stattier \\
Name & Stattier \\
Description & Statistic Tier Client for Random Generator Application \\
Packaging & Jar \\
Java Version & 8 \\
Dependencies & Web, JPA, MySQL, Rest Repositories
\end{tabular}
\end{tabbing}

For the statistic Tier Service we need to create the following Package structure:
\dirtree{%
.1 ch.toky.randgen.statservice.
.2 StatserviceApplication.java.
.2 ch.toky.randgen.statservice.controller.
.3 Controller.java.
.2 ch.toky.randgen.statservice.model.
.3 PodStat.java.
.3 Series.java.
.3 SeriesItem.java.
.3 Stat.java.
.2 ch.toky.randgen.statservice.repository.
.3 PodStatRepository.java.
.2 ch.toky.randgen.statservice.service.
.3 StatService.java.
}

This time, we will create a seperate controller class. Therefore, inside of StatserviceApplication.java will only be the main method which calls the Spring Boot applicaton. Which means we do not have to change this file as Spring Boot already did everything needed for us.

Controller.java will be annotated with @RestController as it contains all our endpoints.

next, we need the StatService class in this controller, this is why we inject it with @Autowired.
\lstinputlisting[language=Java, firstline=16, lastline=17]{Applikationen/statservice/src/main/java/ch/toky/randgen/statservice/controller/Controller.java}
\newpage
Now we can also create our two endpoints, one for all stats, and one with the history.
\lstinputlisting[language=Java, firstline=20, lastline=31]{Applikationen/statservice/src/main/java/ch/toky/randgen/statservice/controller/Controller.java}

We could leave the method out as GET is anyway the default method but we leave it here for for understanding.

Next we are going to declare the Entities and DTOs.

For simplicity i will only show all fields but of cource also the getter and setter methods needs to be implemented.

Fields for PodStat:
\lstinputlisting[language=Java, firstline=11, lastline=16]{Applikationen/statservice/src/main/java/ch/toky/randgen/statservice/model/PodStat.java}

Series will be a DTO and therefore not annotated with Entity we have also a contructor in order to initialize a new Array List.
\lstinputlisting[language=Java, firstline=8, lastline=18]{Applikationen/statservice/src/main/java/ch/toky/randgen/statservice/model/Series.java}

SeriesItem is the second DTO class we are going to use and therefore as well not annotated with @Entity
\lstinputlisting[language=Java, firstline=5, lastline=12]{Applikationen/statservice/src/main/java/ch/toky/randgen/statservice/model/SeriesItem.java}
\newpage
Stat Class:
\lstinputlisting[language=Java, firstline=5, lastline=11]{Applikationen/statservice/src/main/java/ch/toky/randgen/statservice/model/Stat.java}

Now we can implement the Repository (annotated with @Repository) as an interface which extends JpaRepository<PodStat, String>
\lstinputlisting[language=Java, firstline=13, lastline=20]{Applikationen/statservice/src/main/java/ch/toky/randgen/statservice/repository/PodStatRepository.java}

Now the last file is the most important one as there we will have all the business logig in it.
\newpage
We need to annotate it with @Service in order to make it available for Spring Boot to create a Bean out of it.

\lstinputlisting[language=Java, firstline=20, lastline=57]{Applikationen/statservice/src/main/java/ch/toky/randgen/statservice/service/StatService.java}

Now that all the java fils are created we can also create the Dockerfile for this Service.



\subsection{Frontend}
The frontend Application is build with Spring Boot and 

Creating Spring Boot Application:
\begin{tabbing}
\begin{tabular}{ll}
Project & Maven Project \\
Language & Java \\
Spring Boot & 2.1.3 (All versions would apply) \\
Group & ch.toky.randgen \\
Artifact & frontend \\
Name & Frontend \\
Description & Frontend Service \\
Packaging & Jar \\
Java Version & 8 \\
Dependencies & Web, Rest Repositories
\end{tabular}
\end{tabbing}

As we now only have the controller part of our User interface with the Spring boot application, we now need to create the Angular part.

therefore we create a new Project directly with Angular CLI within the folder where all the other Projects are stored.

\begin{lstlisting}[language=Bash]
ng new ui-frontend
\end{lstlisting}

Now we can try if the project has been set up correctly with navigating to the new project folder and hit ng serve.
Once the server is ready we should be able to access localhost:4200 and see a sample home page made by angular cli.

As the frontend should be only one singe application we will now move all folders into the Spring boot application.

in the Folder where all Projects are stored we execute those commands.
\begin{lstlisting}[language=Bash]
mv ui-frontend/src/* frontend/src/
rm -rf ui-frontend/src/
mv ui-frontend/* frontend/
rm -rf ui-frontend
\end{lstlisting}



to install all dependent libraries for 

In order to serve all the Frontend files from the Spring Boot application we need to tell angular to save all compiled files into the target/classes/static folder.
Therefore we need to change the output path option in the file angular.json
\begin{lstlisting}[language=Java]
...
    "outputPath": "target/classes/static",
    ...
\end{lstlisting}

Now every time the Angular Project is built it will be compiled directly to the path where spring boot serves static files.
When we now start the spring boot application we will get the Angular Project under the root path /.

Now we are going to modify the Angular Project itself.

First we create all interfaces with the following commands.
\begin{lstlisting}[language=Bash]
ng generate interface models/randomNumber
ng generate interface models/seriesItem
ng generate interface models/series
ng generate interface models/podStatistic
\end{lstlisting}

Interface RandomNumber
\begin{lstlisting}[language=Java]
id: string;
randNumber: number;
\end{lstlisting}

Interface PodStatistic
\begin{lstlisting}[language=Java]
id: string;
counter: number;
\end{lstlisting}

Interface SeriesItem
\begin{lstlisting}[language=Java]
name: number;
value: number;
\end{lstlisting}

Interface Series
\begin{lstlisting}[language=Java]
name: string;
series: SeriesItem;
\end{lstlisting}

In order to have some empty Objects we also need to declare the following Classes within the folder implementation.

\begin{lstlisting}[language=Bash]
ng generate class RandNumberImpl
ng generate class PodStatisticImpl
\end{lstlisting}
within those classes we define all Strings as empty Strings and all numbers as 0 and all Arrays as empty Arrays [].
this prevents us from getting an error from the Html due to an undefined object.

\begin{lstlisting}[language=Bash]
ng generate service statistic
ng generate service fetch-number
\end{lstlisting}
this created 4 new files for us two typescript and two spec.ts files which would be needed for the end to end tests.

Both Services also need to be declared in the provders array from app.module.ts if not yet done.

We will now modify first the fetch-number service

withing the Constructor argument field we pass a HttpClient.
\begin{lstlisting}[language=Java]
private _http: HttpClient
\end{lstlisting}

Therefore you need to make sure to implement the following Line.
\begin{lstlisting}[language=Java]
import { HttpClient } from '@angular/common/http';
\end{lstlisting}

\begin{lstlisting}[language=Java]
getRandomNumber(): Observable<RandomNumber>{
    return this._http.get<RandomNumber>('randomNumber');
}
\end{lstlisting}

In the Statistic Service File we declare the same HttpClient to be passed to the constructor.
\begin{lstlisting}[language=Java]
private _http: HttpClient
\end{lstlisting}

Then the following methods are going to be declared:
\begin{lstlisting}[language=Java]
getStatistics(): Observable<PodStatistic[]>{

  return this._http.get<PodStatistic[]>('statistics');
}

getHistory(): Observable<Series[]>{

  return this._http.get<Series[]>('history');
}
\end{lstlisting}

As we are using Material design, we now create a seperate Module to manage all depedencies for Material design.

\begin{lstlisting}[language=Bash]
ng generate module material
\end{lstlisting}
Within the newly created material.module.ts file we add the following Modules to imports and exports array.
    MatButtonModule,
    MatCheckboxModule,
    MatCardModule,
    MatTableModule

In order to be able to use our newly created Material Module within our app, we need to add it to the import array from app.module.ts along with all the following imports.
\begin{lstlisting}[language=Java]
	MaterialModule,
    HttpClientModule,
    NgxChartsModule,
    BrowserAnimationsModule
\end{lstlisting}
   
By now we cannot import all the declared Module Dependencies. this is because we use a lot of external libraries which we now have to install via npm.

in order to use Angular Material \footnote{Find documentation via following link \url{https://material.angular.io/guide/getting-started}}
\begin{lstlisting}[language=Bash]
npm install --save @angular/material @angular/cdk @angular/animations 
\end{lstlisting}

We wil use Ngx-Charts Module to display the statistics as a graph.
\footnote{Documentation can be found here \url{https://pixinvent.com/apex-angular-4-bootstrap-admin-template/documentation/documentation-charts-ngx.html}}

and we install the module with the following command.
\begin{lstlisting}[language=Bash]
npm install @swimlane/ngx-charts --save
\end{lstlisting}

Now we can start creating the logig within the Main Component.
first we declatre the Variables which we are going to use.
\begin{lstlisting}[language=Java]
randomNumber: RandomNumber = new RandNumberImpl();
statistics: PodStatistic[] = [new PodStatisticImpl()];
multi: ISeries[];
\end{lstlisting}

In the constructor, we provide our two Services as Private.

next we create the Methods which fetches the data from the service.
\begin{lstlisting}[language=Java]
getNewNumber(){
  this.random_service.getRandomNumber()
  .subscribe(
    data => {this.randomNumber = data;});
}
  
getStatistics(){
  this.statistic_service.getStatistics()
  .subscribe(
     data => {
     this.dataSource = new MatTableDataSource<PodStatistic>(data);});
}

getHistory(){
  this.statistic_service.getHistory()
  .subscribe(
    data =>{this.multi = data;});
}
\end{lstlisting}

We also need to define a method Called "ButtonClick" as we want to use it from the Html file. within this Method we will do nothing more that just calling all fetch methods.
\begin{lstlisting}[language=Java]
buttonClick(){
    this.getNewNumber();
    this.getStatistics();
    this.getHistory();
}
\end{lstlisting}

from within ngOnInit we call only buttonClick().

As a last thing in this file we need to add some configuration Variables.
\begin{lstlisting}[language=Java]
// Table
displayedColumns: string[] = ['id','counter'];
dataSource: any;

// Graph
view: any[] = [700, 400];
showXAxis = true;
showYAxis = true;
gradient = false;
showLegend = true;
showXAxisLabel = true;
xAxisLabel = 'Sum of all Calls';
showYAxisLabel = true;
yAxisLabel = 'Calls per Pod';
timeline = true;
colorScheme = {
 domain: ['#5AA454', '#A10A28', '#C7B42C', '#AAAAAA']
};
\end{lstlisting}

the Last file we need to create for the Frontend is the HTML from our component Class.
\begin{lstlisting}[language=HTML]
<section class="mat-typography">

<h1>Random Numbers</h1>

<mat-card>
  <mat-card-header>
      <p>Id: {{randomNumber.id}}</p>
  </mat-card-header>
  <mat-card-content>
      <p>Number: {{randomNumber.randNumber}}</p>
  </mat-card-content>

</mat-card>

  <button mat-button (click)="buttonClick()">Get New Number</button>

<mat-card>
  <mat-card-header>
    Statistics Table
  </mat-card-header>
  <mat-card-content >
    <table mat-table [dataSource]="dataSource" class="mat-elevation-z8">
      <ng-container matColumnDef="id">
        <th mat-header-cell *matHeaderCellDef>ID</th>
        <td mat-cell *matCellDef="let element">{{element.id}}</td>
      </ng-container>
      <ng-container matColumnDef="counter">
        <th mat-header-cell *matHeaderCellDef>Counter</th>
        <td mat-cell *matCellDef="let element">{{element.counter}}</td>
      </ng-container>
      <tr mat-header-row *matHeaderRowDef="displayedColumns"></tr>
      <tr mat-row *matRowDef="let row; columns: displayedColumns;"></tr>
    </table>
  </mat-card-content>
</mat-card>

    <ngx-charts-line-chart 
    [view]="view"
    [scheme]="colorScheme"
    [results]="multi"
    [gradient]="gradient"
    [xAxis]="showXAxis"
    [yAxis]="showYAxis"
    [legend]="showLegend"
    [showXAxisLabel]="showXAxisLabel"
    [showYAxisLabel]="showYAxisLabel"
    [xAxisLabel]="xAxisLabel"
    [yAxisLabel]="yAxisLabel"
    [autoScale]="autoScale"
    [timeline]="timeline"
    (select)="onSelect($event)">
    ></ngx-charts-line-chart>


</section>
\end{lstlisting}

\subsection{running with Docker}

\textbf{Create a Network}
\begin{lstlisting}[language=Bash]
docker network create  -d bridge rand-gen-network
\end{lstlisting}

\textbf{Start the Database Container}

\begin{lstlisting}[language=Bash]
docker run -d --name rand-gen-database \
-e MYSQL_ROOT_PASSWORD='password' \
-e MYSQL_USER='random_user' \
-e MYSQL_PASSWORD='random_password' \
-e MYSQL_DATABASE='rand_numbers' \
--network rand-gen-network \
mysql:5.6
\end{lstlisting}

\textbf{Start the Random Generator App}
\begin{lstlisting}[language=Bash]
docker run -d --name=rand-gen-app \
--network rand-gen-network \
rand-gen-image
\end{lstlisting}

\textbf{Start the middle tier Application}
\begin{lstlisting}[language=Bash]
docker run -d \
--name middle-tier \
--network rand-gen-network \
middle-tier
\end{lstlisting}

\textbf{Starting the Stat Tier Application}
\begin{lstlisting}[language=Bash]
docker run -d \
--name stat-tier \
--network rand-gen-network \ 
stat-tier-image
\end{lstlisting}

\textbf{Start the Frontend Service}
\begin{lstlisting}[language=Bash]
docker run -d -p 8080:8080 \
--network rand-gen-network \
--name rand-frontend \
rand-frontend
\end{lstlisting}