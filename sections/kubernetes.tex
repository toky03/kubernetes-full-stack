\subsection{Installing Kubernetes}

To play arround and get to know Kubernetes it is recommendet to install minikube \footnote{\url{https://kubernetes.io/docs/tasks/tools/install-minikube/}} which is running inside a VM on your local machine.
If you want to use Kubernetes on a Server and deploy PROD ready application then you have the option to rent directly a Kubernetes Cluster from a Vendor
\begin{itemize}
\item AWS EKS \url{https://aws.amazon.com/de/eks/}
\item GKE Google Cloud \url{https://cloud.google.com/kubernetes-engine/}
\item IBM Kubernetes Service \url{https://www.ibm.com/cloud/container-service}
\end{itemize}

You can also install Kubernetes on a plain server, we use ubuntu for this setup. weh use the following instruction to get this done \footnote{\url{https://linuxconfig.org/how-to-install-kubernetes-on-ubuntu-18-04-bionic-beaver-linux}}
 But first we need to add the Kubernetes to our installation repository.
\begin{verbatim}
curl -s https://packages.cloud.google.com/apt/doc/apt-key.gpg | sudo apt-key add
\end{verbatim}

\begin{verbatim}
sudo apt-add-repository "deb http://apt.kubernetes.io/ kubernetes-xenial main"
$ sudo apt install kubeadm 
\end{verbatim}

\begin{verbatim}
sudo swapoff -a
\end{verbatim}

Now we have Kubeadm installed, we can continue with the server setup based on the instruction from Kubernetes itself \footnote{\url{https://kubernetes.io/docs/setup/independent/create-cluster-kubeadm/}}

The pod network will be Callico.

\begin{verbatim}
kubeadm init --apiserver-advertise-address=<ip-address> --pod-network-cidr=192.168.0.0/16
\end{verbatim}

Switch back to non root user and execute those commands.

\begin{verbatim}
mkdir -p $HOME/.kube
sudo cp -i /etc/kubernetes/admin.conf $HOME/.kube/config
sudo chown $(id -u):$(id -g) $HOME/.kube/config
\end{verbatim}

/proc/sys/net/bridge/bridge-nf-call-iptables to 1

Now we just need to deploy the pod network.
\begin{verbatim}
kubectl apply -f https://docs.projectcalico.org/v3.3/getting-started/kubernetes/installation/hosted/rbac-kdd.yaml
kubectl apply -f https://docs.projectcalico.org/v3.3/getting-started/kubernetes/installation/hosted/kubernetes-datastore/calico-networking/1.7/calico.yaml
\end{verbatim}

Wait until all pods are up and running.
kubectl get pods --all-namespaces

If we are using a single node cluster, we need to tell Kubernetes that it is allowed to deploy pods on the master node (which is the only node in the cluster)
kubectl taint nodes --all node-role.kubernetes.io/master-

Now we are done with the setup and we can start deployin pods to the Kubernetes cluster.

\subsection{Deploy Services}
All the Applications are running inside of a docker container and managed within a Kubernetes Pod and accessible via a Kubernetes Service where only the Frontend Service is bound to a Node Port.

\textbf{Database}
A database is always quite special for a deployment as the whole Container setup is meant to run stateless. Which means all the data is lost once a container dies. Mostly this is very bad for a database as there all the relevant data should be persisted.

Docker itself provides therefore persistent Volumens which can be mounted by a Container.

Within Kubernetes we create therefore a PersistentVolumeClaim \footnote{\url{https://kubernetes.io/docs/concepts/storage/persistent-volumes/}} before we start a 

Deployment
\lstinputlisting[language=Bash,firstline=16]{config-files/mysql-deployment.yml}

As a pod can always vary its ip adress, we need to have a stable endpoint to access the pods.
Specificly for this purpose, kubernetes provides services.
Service to make the Database Pod available 
\lstinputlisting[language=Bash,firstline=1,lastline=14]{config-files/mysql-deployment.yml}

\textbf{Random Generator}
Now we also declare the deployment for the generator service.
Notice that we have two replicas for the generator. this is because we wand to have two diffrent ids as a source for our generator.
\lstinputlisting[language=Bash,firstline=17]{config-files/rand-gen-deployment.yml}

The corresponsing service.
\lstinputlisting[language=Bash,lastline=14]{config-files/rand-gen-deployment.yml}

\textbf{Middle Tier service}
\lstinputlisting[language=Bash,firstline=19]{config-files/middle-tier-deployment.yml}
The service
\lstinputlisting[language=Bash,lastline=16]{config-files/middle-tier-deployment.yml}

\textbf{Statistic Service}
\lstinputlisting[language=Bash,firstline=19]{config-files/stat-tier-deployment.yml}
Service
\lstinputlisting[language=Bash,lastline=17]{config-files/stat-tier-deployment.yml}

\textbf{Frontend}
\lstinputlisting[language=Bash,firstline=18]{config-files/frontend-deployment.yml}

Service:
\lstinputlisting[language=Bash,lastline=15]{config-files/frontend-deployment.yml}
