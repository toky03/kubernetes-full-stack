The frontend Application is build with Spring Boot and 

Creating Spring Boot Application:
\begin{tabbing}
\begin{tabular}{ll}
Project & Maven Project \\
Language & Java \\
Spring Boot & 2.1.3 (All versions would apply) \\
Group & ch.toky.randgen \\
Artifact & frontend \\
Name & Frontend \\
Description & Frontend Service \\
Packaging & Jar \\
Java Version & 8 \\
Dependencies & Web, Rest Repositories
\end{tabular}
\end{tabbing}

As we now only have the controller part of our User interface with the Spring boot application, we now need to create the Angular part.

therefore we create a new Project directly with Angular CLI within the folder where all the other Projects are stored.

\begin{lstlisting}[language=Bash]
ng new ui-frontend
\end{lstlisting}

Now we can try if the project has been set up correctly with navigating to the new project folder and hit ng serve.
Once the server is ready we should be able to access localhost:4200 and see a sample home page made by angular cli.

As the frontend should be only one singe application we will now move all folders into the Spring boot application.

in the Folder where all Projects are stored we execute those commands.
\begin{lstlisting}[language=Bash]
mv ui-frontend/src/* frontend/src/
rm -rf ui-frontend/src/
mv ui-frontend/* frontend/
rm -rf ui-frontend
\end{lstlisting}



to install all dependent libraries for 

In order to serve all the Frontend files from the Spring Boot application we need to tell angular to save all compiled files into the target/classes/static folder.
Therefore we need to change the output path option in the file angular.json
\begin{lstlisting}[language=Java]
...
    "outputPath": "target/classes/static",
    ...
\end{lstlisting}

Now every time the Angular Project is built it will be compiled directly to the path where spring boot serves static files.
When we now start the spring boot application we will get the Angular Project under the root path /.

Now we are going to modify the Angular Project itself.

First we create all interfaces with the following commands.
\begin{lstlisting}[language=Bash]
ng generate interface models/randomNumber
ng generate interface models/seriesItem
ng generate interface models/series
ng generate interface models/podStatistic
\end{lstlisting}

Interface RandomNumber
id: string;
randNumber: number;

Interface PodStatistic
id: string;
counter: number;

Interface SeriesItem
name: number;
value: number;

Interface Series
name: string;
series: SeriesItem;


\begin{lstlisting}[language=Bash]
ng generate service statistic
ng generate service fetch-number
\end{lstlisting}
this created 4 new files for us two typescript and two spec.ts files which would be needed for the end to end tests.

Both Services also need to be declared in the provders array from app.module.ts if not yet done.

We will now modify first the fetch-number service

withing the Constructor argument field we pass a HttpClient.
private _http: HttpClient

Therefore you need to make sure to implement the following Line.
import { HttpClient } from '@angular/common/http';

\begin{lstlisting}[language=Java]
getRandomNumber(): Observable<RandomNumber>{
    return this._http.get<RandomNumber>('randomNumber');
}
\end{lstlisting}

In the Statistic Service File we declare the same HttpClient to be passed to the constructor.
private _http: HttpClient

Then the following methods are going to be declared:
\begin{lstlisting}[language=Java]
getStatistics(): Observable<PodStatistic[]>{

  return this._http.get<PodStatistic[]>('statistics');
}

getHistory(): Observable<Series[]>{

  return this._http.get<Series[]>('history');
}
\end{lstlisting}

As we are using Material design, we now create a seperate Module to manage all depedencies for Material design.

\begin{lstlisting}[language=Bash]
ng generate module material
\end{lstlisting}
Within the newly created material.module.ts file we add the following Modules to imports and exports array.
    MatButtonModule,
    MatCheckboxModule,
    MatCardModule,
    MatTableModule

In order to be able to use our newly created Material Module within our app, we need to add it to the import array from app.module.ts along with all the following imports.
	MaterialModule,
    HttpClientModule,
    NgxChartsModule,
    BrowserAnimationsModule
   
By now we cannot import all the declared Module Dependencies. this is because we use a lot of external libraries which we now have to install via npm.

in order to use Angular Material \footnote{Find documentation via following link \url{https://material.angular.io/guide/getting-started}}
\begin{lstlisting}[language=Bash]
npm install --save @angular/material @angular/cdk @angular/animations 
\end{lstlisting}

We wil use Ngx-Charts Module to display the statistics as a graph.
\footnote{Documentation can be found here \url{https://pixinvent.com/apex-angular-4-bootstrap-admin-template/documentation/documentation-charts-ngx.html}}

and we install the module with the following command.
\begin{lstlisting}[language=Bash]
npm install @swimlane/ngx-charts --save
\end{lstlisting}