The frontend Application is build with Spring Boot and Angular together.

Creating Spring Boot Application:
\begin{tabbing}
\begin{tabular}{ll}
Project & Maven Project \\
Language & Java \\
Spring Boot & 2.1.3 (All versions would apply) \\
Group & ch.toky.randgen \\
Artifact & frontend \\
Name & Frontend \\
Description & Frontend Service \\
Packaging & Jar \\
Java Version & 8 \\
Dependencies & Web, Rest Repositories
\end{tabular}
\end{tabbing}

Until now we only have the controller part of our User interface with the Spring boot application, we need to create the Angular partnow.

Therefore, we create a new Project directly with Angular CLI within the folder where all the other Projects are stored.

\begin{verbatim}
ng new ui-frontend
\end{verbatim}

Now to try if the project has been set up correctly. Change path to the new project folder and run \texttt{ng serve}.
Once the server is ready we should be able to access http://http://localhost:4200 and see a sample home page created by angular cli.

As the frontend should be only one single application we will now move all folders into the Spring boot application folder.

in the Folder where all Projects are stored we execute those commands.
\begin{verbatim}
mv ui-frontend/src/* frontend/src/
rm -rf ui-frontend/src/
mv ui-frontend/* frontend/
rm -rf ui-frontend
\end{verbatim}

In order to serve all the frontend files from the Spring Boot application we need to tell angular to save all compiled files into the target/classes/static folder.
In order to achieve this, we need to change the output path option in the file angular.json
\lstinputlisting[language=Java, firstline=15, lastline=17]{Applikationen/frontend/angular.json}

Now every time the Angular Project is built with \texttt{ng build} it will be compiled and copied directly to the path where spring boot serves static files.
Every time the spring boot application has started we will get the static web files under the root path /.

Now we are going to modify the Angular Project itself.

First we create all interfaces with the following commands.
\begin{verbatim}
ng generate interface models/randomNumber
ng generate interface models/seriesItem
ng generate interface models/series
ng generate interface models/podStatistic
\end{verbatim}

Within Interface RandomNumber, declare those two fields.
\lstinputlisting[language=Java, firstline=2, lastline=3]{Applikationen/frontend/src/app/models/random-number.ts}

Within Interface PodStatistic
\lstinputlisting[language=Java, firstline=2, lastline=3]{Applikationen/frontend/src/app/models/pod-statistic.ts}

Interface SeriesItem
\lstinputlisting[language=Java, firstline=2, lastline=3]{Applikationen/frontend/src/app/models/series-item.ts}

Interface Series
\lstinputlisting[language=Java, firstline=4, lastline=5]{Applikationen/frontend/src/app/models/series.ts}

In order to have some empty Objects we also need to declare the following Classes within the folder implementation.

\begin{verbatim}
ng generate class RandNumberImpl
ng generate class PodStatisticImpl
\end{verbatim}
Where RandomNumberImpl implements RandomNumber and PodStatisticImpl implements PodStatistic
Within those classes we define all Strings as empty Strings and all numbers as 0.
This prevents us from getting an error from the Html due to an undefined object.

\begin{verbatim}
ng generate service statistic
ng generate service fetch-number
\end{verbatim}
this created 4 new files for us two typescript and two spec.ts files which would be needed for the end to end tests.
Both Services also need to be declared in the provders array from $app.module.ts$ if not yet done from angular cli.

We will now modify first the fetch-number service

withing the Constructor argument field we pass a HttpClient.
\lstinputlisting[language=Java, firstline=11, lastline=12]{Applikationen/frontend/src/app/fetch-number.service.ts}

Therefore you need to make sure to implement the following Line.
\lstinputlisting[language=Java, firstline=2, lastline=2]{Applikationen/frontend/src/app/fetch-number.service.ts}
\newpage
\lstinputlisting[language=Java, firstline=13, lastline=15]{Applikationen/frontend/src/app/fetch-number.service.ts}

In the Statistic Service File we declare the same HttpClient to be passed to the constructor.
\lstinputlisting[language=Java, firstline=12, lastline=13]{Applikationen/frontend/src/app/statistic.service.ts}

Then the following methods are going to be declared:
\lstinputlisting[language=Java, firstline=14, lastline=22]{Applikationen/frontend/src/app/statistic.service.ts}

As we are using Material design, we need to create a seperate Module to manage all depedencies which are used for Material design.

\begin{verbatim}
ng generate module material
\end{verbatim}
Within the newly created $material.module.ts$ file we add the following Modules into imports and exports array.
\lstinputlisting[language=Java, firstline=8, lastline=12]{Applikationen/frontend/src/app/material/material.module.ts}
In order to be able to use our newly created Material Module within our app, we need to add it to the import array from $app.module.ts$ along with all the following imports.
\lstinputlisting[language=Java, firstline=16, lastline=20]{Applikationen/frontend/src/app/app.module.ts}
   
By now we cannot import all the declared Module Dependencies. this is because we use a lot of external libraries which we now have to install via npm.

in order to use Angular Material \footnote{documentation via following link \url{https://material.angular.io/guide/getting-started}}
\begin{verbatim}
npm install --save @angular/material \\
@angular/cdk @angular/animations 
\end{verbatim}
\newpage
Ngx-Charts Module is to display the statistics as a graph.
\footnote{Documentation can be found here \url{https://swimlane.gitbook.io/ngx-charts/}}

It can be installed with the following command.
\begin{verbatim}
npm install @swimlane/ngx-charts --save
\end{verbatim}

Now we can start adding the business logic.
Within the Main Component file $app.component.ts$.
First we declatre the Variables which we are going to use.
\lstinputlisting[language=Java, firstline=19, lastline=23]{Applikationen/frontend/src/app/app.component.ts}

In the constructor, we provide our two services as Private.

next we create the Methods which fetches the data from the service.
\lstinputlisting[language=Java, firstline=45, lastline=62]{Applikationen/frontend/src/app/app.component.ts}

We also need to define a method Called $ButtonClick$ as we want to use it from the Html file. within this Method we will do nothing more that just calling all fetch methods.
\lstinputlisting[language=Java, firstline=64, lastline=68]{Applikationen/frontend/src/app/app.component.ts}

From $ngOnInit$ we call only $buttonClick()$.

then we also have a onSelect which we need ot provice for the graph;
\lstinputlisting[language=Java, firstline=74, lastline=76]{Applikationen/frontend/src/app/app.component.ts}
\newpage
As a last thing in this file we need to add some configuration Variables.

\lstinputlisting[language=Java, firstline=25, lastline=39]{Applikationen/frontend/src/app/app.component.ts}

the Last file we need to create for the Frontend is the html from our component Class.
\lstinputlisting[language=HTML]{Applikationen/frontend/src/app/app.component.html}

Now it is time to go through the Server side Part of the Frontend, which is from our Spring Boot frontend.
Like we did it for all the previous Spring boot Applications, we will first create all folders and .java files.
From within src/main/java
\dirtree{%
.1 ch.toky.randgen.frontend.
.2 FrontendApplication.java.
.2 ch.toky.randgen.frontend.model.
.3 PodStat.java.
.3 RandomNumber.java.
.3 Series.java.
.3 SeriesItem.java.
.2 ch.toky.randgen.frontend.service.
.3 RestService.java.
}
All classes under the Model Package, represent our DTOs like we had it for all the previous services.

Fields from PodStat:
\lstinputlisting[language=Java, firstline=5, lastline=8]{Applikationen/frontend/src/main/java/ch/toky/randgen/frontend/model/PodStat.java}

Fields from RandomNumber:
\lstinputlisting[language=Java, firstline=5, lastline=6]{Applikationen/frontend/src/main/java/ch/toky/randgen/frontend/model/RandomNumber.java}

Series:
\lstinputlisting[language=Java, firstline=7, lastline=8]{Applikationen/frontend/src/main/java/ch/toky/randgen/frontend/model/Series.java}

SeriesItem:
\lstinputlisting[language=Java, firstline=5, lastline=6]{Applikationen/frontend/src/main/java/ch/toky/randgen/frontend/model/SeriesItem.java}

All fields from above classes must have as well their coresponsive getter and setter methods.

To fetch all the Data we need to declare witin $RestService.java$ all the methods which communicates to Statistic Service and to Middle Tier Service.

As this service should be injected within Main Application we need to annotate this Class with @Service.

Fields from RestService:
\lstinputlisting[language=Java, firstline=16, lastline=20]{Applikationen/frontend/src/main/java/ch/toky/randgen/frontend/service/RestService.java}

As one can see, we will retreive all endpoints via environment Variables which are either declared directly within the docker Container or via Kubernetes Environment Variable configurations.

Now we declare the methods.
\lstinputlisting[language=Java, firstline=22, lastline=34]{Applikationen/frontend/src/main/java/ch/toky/randgen/frontend/service/RestService.java}

This was the last method for  $RestService.java$.
Within the Main Application file FrontendApplication.java we will also declare our endpoints we provide to the Angular application. Therefore we need to add the @RestController annotation besides the @SpringBootApplication which should already be in place.
\newpage
There is one dependency with @Autowired and this is our previously declared RestService.
\lstinputlisting[language=Java, firstline=23, lastline=24]{Applikationen/frontend/src/main/java/ch/toky/randgen/frontend/FrontendApplication.java}


Now declare all endpoint methods.
\lstinputlisting[language=Java, firstline=26, lastline=37]{Applikationen/frontend/src/main/java/ch/toky/randgen/frontend/FrontendApplication.java}

Within the root directory of the project we can place the Docker file like we did it for all the other services. Except that we now have to add another stage for the angular build.





