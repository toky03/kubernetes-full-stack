The Statistic Service has the same setup as the Middle Tier client.
Therefore the Project will be created too with Spring Boot initializer.
\begin{tabbing}
\begin{tabular}{ll}
Project & Maven Project \\
Language & Java \\
Spring Boot & 2.1.3 (All versions would apply) \\
Group & ch.toky.randgen \\
Artifact & stattier \\
Name & Stattier \\
Description & Statistic Tier Client for Random Generator Application \\
Packaging & Jar \\
Java Version & 8 \\
Dependencies & Web, JPA, MySQL, Rest Repositories
\end{tabular}
\end{tabbing}

For the statistic Tier Service we need to create the following Package structure:
\dirtree{%
.1 ch.toky.randgen.statservice.
.2 StatserviceApplication.java.
.2 ch.toky.randgen.statservice.controller.
.3 Controller.java.
.2 ch.toky.randgen.statservice.model.
.3 PodStat.java.
.3 Series.java.
.3 SeriesItem.java.
.3 Stat.java.
.2 ch.toky.randgen.statservice.repository.
.3 PodStatRepository.java.
.2 ch.toky.randgen.statservice.service.
.3 StatService.java.
}

This time, we will create a seperate controller class. Therefore, inside of StatserviceApplication.java will only be the main method which calls the Spring Boot applicaton. Which means we do not have to change this file as Spring Boot already did everything needed for us.

Controller.java will be annotated with @RestController as it contains all our endpoints.

next, we need the StatService class in this controller, this is why we inject it with @Autowired.
\lstinputlisting[language=Java, firstline=16, lastline=17]{Applikationen/statservice/src/main/java/ch/toky/randgen/statservice/controller/Controller.java}
\newpage
Now we can also create our two endpoints, one for all stats, and one with the history.
\lstinputlisting[language=Java, firstline=20, lastline=31]{Applikationen/statservice/src/main/java/ch/toky/randgen/statservice/controller/Controller.java}

We could leave the method out as GET is anyway the default method but we leave it here for for understanding.

Next we are going to declare the Entities and DTOs.

For simplicity i will only show all fields but of cource also the getter and setter methods needs to be implemented.

Fields for PodStat:
\lstinputlisting[language=Java, firstline=11, lastline=16]{Applikationen/statservice/src/main/java/ch/toky/randgen/statservice/model/PodStat.java}

Series will be a DTO and therefore not annotated with Entity we have also a contructor in order to initialize a new Array List.
\lstinputlisting[language=Java, firstline=8, lastline=18]{Applikationen/statservice/src/main/java/ch/toky/randgen/statservice/model/Series.java}

SeriesItem is the second DTO class we are going to use and therefore as well not annotated with @Entity
\lstinputlisting[language=Java, firstline=5, lastline=12]{Applikationen/statservice/src/main/java/ch/toky/randgen/statservice/model/SeriesItem.java}
\newpage
Stat Class:
\lstinputlisting[language=Java, firstline=5, lastline=11]{Applikationen/statservice/src/main/java/ch/toky/randgen/statservice/model/Stat.java}

Now we can implement the Repository (annotated with @Repository) as an interface which extends JpaRepository<PodStat, String>
\lstinputlisting[language=Java, firstline=13, lastline=20]{Applikationen/statservice/src/main/java/ch/toky/randgen/statservice/repository/PodStatRepository.java}

Now the last file is the most important one as there we will have all the business logig in it.
\newpage
We need to annotate it with @Service in order to make it available for Spring Boot to create a Bean out of it.

\lstinputlisting[language=Java, firstline=20, lastline=57]{Applikationen/statservice/src/main/java/ch/toky/randgen/statservice/service/StatService.java}

Now that all the java fils are created we can also create the Dockerfile for this Service.

