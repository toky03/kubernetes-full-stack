The Random Generator will be a verry simple Python App which has an unique Id per Service instance. It will return his id and a new Random Number. For this application we create a folder called RandGen with the following Structure.

\dirtree{%
.1 RandGen.
.2 Dockerfile.
.2 rand\_gen.py.
.2 requirements.txt.
}
The Dockerfile is needed to create the Docker Image which will be used from Kubernetes to Create the Random Generator Service.
rand\_gen.py is the complete Random Generator Python application based on Flask\footnote{More information about Flask can be found on the official Homepage \url{http://flask.pocoo.org/}}.
 \lstinputlisting[language=Python]{Applikationen/generator/rand_gen.py}
In requirements.txt are the Python packages specified. In this case it is flask. therefore this file contains only one line which is
\begin{verbatim}
Flask==1.0.2
\end{verbatim}

The Dockerfile to create the Docker image:
\lstinputlisting[language=Bash]{Applikationen/generator/Dockerfile}
